具体例を出したい。
\(f : A \to B\) とする。
\(P , Q\) : \(A , B\) 上の述語に対して \(P(x)\) が成り立つなら \(Q(f(x))\) が成り立つとする。
このとき \(f : \texttt{Refined } A \texttt{ with } P \to \texttt{Refined } B \texttt{ with } Q \) と型付けられるはず。
仮定としては、
  \[\Gamma := A : \texttt{Type} , B : \texttt{Type} , f : A \to B , P : A \to \texttt{Prop} , Q : A \to \texttt{Prop}\]
をまず考え、これに命題に対応する項がくっつく。
\(K := \texttt{Forall} \ x \ A \ \{(\texttt{Apply} \ P \ x) \to (\texttt{Apply} \ Q \ (\texttt{Apply} \ f \ x)))\}\) なる \(\texttt{Prop}\) 型の項が 「 \(P(x)\) が成り立つなら \(\ldots\) 」に対応する。
改めて \(\Gamma \leftarrow \Gamma , \text{Hold } K\) と置き直して、 \(\Gamma \vdash f : \texttt{Ref} \ A \ P \to \texttt{Ref} \ B \ Q\) を示す。

\[
  \infer[\text{conversion}]
  {\Gamma \vdash f : \texttt{Ref} \ A \ P \to \texttt{Ref} \ B \ Q}{
    \infer[\text{forall elim}]{\Gamma \vdash \texttt{Fun} \ x \ (\texttt{Apply} \ f \ x) : \texttt{Ref} \ A \ P \to \texttt{Ref} \ B \ Q}{
      \infer[]{\Gamma , x : \texttt{Ref} \ A \ P \vdash \texttt{Apply} \ f \ x : \texttt{Ref} \ B \ Q}{
        \infer[]{}{}
      }
  }}
\]