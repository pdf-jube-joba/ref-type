証明項に対応するものを微妙に導入して、型システムが自動的に判定できる証明を増やした形。

\paragraph*{項やコンテキストの定義}
\begin{grammar}
<term> ::= <variable> 
\alt `Prop'
\alt `Type'
\alt `Fun' <variable> <term> <term>
\alt `For' <variable> <term> <term>
\alt `App' <term> <term>
\alt `Ref' <term> <term>
\alt `Prf' <term>

<context-snippet> ::= <variable> `:' <term>
\alt Hold <term>

<context> ::= `empty' | <context> `,' <context-snippet>
\end{grammar}

\texttt{Proof} は証明項を陽に扱うためのもの。
命題 \(P\) が示せるときに \(\texttt{Proof } P\) を証明項として扱ってよいとすることで簡単にならないだろうか?
あるいは、
(証明が必要となるところでのみ用いたい)

\paragraph*{項やコンテキストの評価}
ここでコンテキストや項の関係を定義していく。

\begin{itembox}[l]{コンテキストと項の関係}
  \(\vdash \Gamma\) , コンテキストの well-def 性 \\
  \(\Gamma \vdash t_1 : t_2\) , 項の型付け性 \\
  \(\Gamma \vDash t\) , 項の証明可能性
\end{itembox}

\begin{itembox}[l]{コンテキストの well-def}
  \[\infer[\text{context empty}]{\vdash empty}{}\]
  \[
    \infer[\text{context type}]{\vdash \Gamma , x : \text{Sort}}{\vdash \Gamma & x \notin \text{FV}(\Gamma)}\,
    \infer[\text{context term}]{\vdash \Gamma , x : A}{\vdash \Gamma & \Gamma \vdash A : \texttt{Sort} & x \notin \text{FV}(\Gamma)}\]
  \[\infer[\text{context prop}]{\vdash \Gamma , \text{Hold } P}{\vdash \Gamma & \Gamma \vdash P : \texttt{Prop}}\]
\end{itembox}

\begin{itembox}[l]{コンテキスト 自明}
  \[
    \infer[\text{variable-type}]{\Gamma , x : \texttt{Sort} \vdash x : \texttt{Sort}}{\vdash \Gamma & x \notin \text{FV}(\Gamma)} \,
    \infer[\text{variable-term}]{\Gamma , x : A \vdash x : A}{\vdash \Gamma & \Gamma \vdash A : \texttt{Type} & x \notin \text{FV}(\Gamma)}\]
  \[\infer[\text{weakning}]{\Gamma , \_ \vdash t : A_2}{\vdash \Gamma , \_ & \Gamma \vdash t : A_2} \]
\end{itembox}

\begin{itembox}[l]{formation}
  \[\infer[\text{forall formation(type-type)}]{\Gamma \vdash \texttt{For} \, x \, A_1 \, A_2 : \texttt{Type}}{\Gamma \vdash A_1 : \texttt{Type} & \Gamma , x : A_1 \vdash A_2 : \texttt{Type} }\]
  \[\infer[\text{forall formation(type-prop)}]{\Gamma \vdash \texttt{For} \, x \, A_1 \, A_2 : \texttt{Prop}}{\Gamma \vdash A_1 : \texttt{Type} & \Gamma , x : A_1 \vdash A_2 : \texttt{Prop} }\]
  \[\infer[\text{forall formation(prop-prop)}]{\Gamma \vdash \texttt{For} \, x \, A_1 \, A_2 : \texttt{Prop}}{\Gamma \vdash A_1 : \texttt{Prop} & \Gamma , x : A_1 \vdash A_2 : \texttt{Prop} }\]
  \[\infer[\text{refinement formation}]{\Gamma \vdash \texttt{Ref} \, A \, P : \texttt{Type}}{\Gamma \vdash A : \texttt{Type} & \Gamma \vdash P : \texttt{For} \, x \, A \, \texttt{Prop}}\]
  Prop の項から Type つくるの禁止(型で制限)
\end{itembox}

\begin{itembox}[l]{\(\beta\) 同値について}
  \[\infer[\text{conversion}]{\Gamma \vdash x : A_2}{\Gamma \vdash x : A_1 & A_1 \equiv_{\beta} A_2 & \Gamma \vdash A_2 : \texttt{Sort}} \]
  \[\infer[\text{conversion(type)}]{\Gamma \vdash A_2 : \texttt{Type}}{\Gamma \vdash A_1 : \texttt{Type} & A_1 \equiv_{\beta} A_2} \,
  \infer[\text{conversion(prop)}]{\Gamma \vdash A_2 : \texttt{Prop}}{\Gamma \vdash A_1 : \texttt{Prop} & A_1 \equiv_{\beta} A_2}\]
  下二つは application と conversion があればいらないかもしれない。
  必要になったら足す感じで。
\end{itembox}

\begin{itembox}[l]{introduction と elimination}
  \[
    \infer[\text{for intro}]{\Gamma \vdash \texttt{Fun} \, x \, A_1 \, t : \texttt{For} \, x \, A_1 \, A_2}{\Gamma , x : A_1 \vdash t : A_2 & \Gamma \vdash \texttt{For} \, x \, A_1 \, A_2 : \texttt{Type}} \,
    \infer[\text{for elim}]{\Gamma \vdash \texttt{App} \, t_1 \, t_2 : A_2\{x \leftarrow t_2\}}{\Gamma \vdash t_1 : \texttt{For} \, x \, A_1 \, A_2 & \Gamma \vdash t_2 : A_1} \]
  \[
    \infer[\text{ref intro}]{\Gamma \vdash t : \texttt{Ref} \, A \, P}{\Gamma \vdash t : A & \Gamma \vdash \texttt{Ref} \, A \, P : \texttt{Type} & \Gamma \vDash \texttt{App} \, P \, t} \,
    \infer[\text{ref elim}]{\Gamma \vdash t : A}{\Gamma \vdash t : \texttt{Ref} \, A \, P} \]
\end{itembox}

\begin{itembox}[l]{proof term について}
  \[
    \infer[\text{assumption}]{\Gamma , \text{Hold} \, P \vDash P}{}\,
    \infer[\text{weakning}]{\Gamma , \_ \vDash P}{\Gamma \vDash P}
  \]
  \[
    \infer[\text{implicit proof}]{\Gamma \vDash P}{\Gamma \vdash P : \texttt{Prop} & \Gamma \vdash t : P} \,
    \infer[\text{explicit proof}]{\Gamma \vdash \texttt{Prf} \, P : P}{\Gamma \vdash P : \texttt{Prop} &\Gamma \vDash P} \]
  \[
    \infer[\text{refinement inversion}]{\Gamma \vDash \texttt{App} \, P \, t}{\Gamma \vdash t : \texttt{Ref} \, A \, P}\]
  \[
    \infer[\text{proof conversion}]{\Gamma \vDash P_2}{\Gamma \vDash P_1 & P_1 \equiv_{\beta} P_2} \]
  Proof conversion は多分いらない。
  explicit proof と conversion によって Proof \(P_1\) が Proof \(P_1\) : \(P_2\) と型付けされる。
\end{itembox}