ラムダ項の定義としては
\begin{itembox}[l]{項の定義}
  \begin{grammar}
    <term ; t> ::= <variable ; x>
    \alt `Fun' <variable> <term>
    \alt `App' <term> <term>
  \end{grammar}
\end{itembox}

これについては型理論はないが、
ラムダ項にどういうものが定義されていたかと言うと、
\(\alpha\) 変換(ラムダ項の間の関係かラムダ項の変換)・ \(\beta\) 変換(ラムダ項の間の関係かラムダ項の変換)・ reduction (ラムダ項の間の関係)・値(ラムダ項の性質)・正規形(ラムダ項の性質)などがあった。
合流性やらが成り立つのだった。
項 \(t_1\{x \leftarrow t_2\}\) のような代入が定義されている。
項の間の同値関係 \(t_1 \equiv t_2\) を次の規則から生成する。

\begin{itembox}[l]{ラムダ項の等式}
  \[\infer[\text{alpha}]{\texttt{Fun} \, x_1 \, t_1 \equiv \texttt{Fun} \, x_2 \, (t_1\{x_1 \leftarrow t_2\})}{x_2 \notin \text{FV}(t_2)}\]
  \[\infer[\text{computation}]{\texttt{App} \, (\texttt{Fun} \, x \, t_1) \, t_2 \equiv t_2\{x \leftarrow t_2\}}{x \notin \text{FV}(t_2)}\]
  \[\infer[\text{conversion-fun}]{\texttt{Fun} \, x \, t_1 \equiv \texttt{Fun} \, x \, t_2}{t_1 \equiv t_2}\]
  \[\infer[\text{conversion-app-1}]{\texttt{App} \, t_1 \, t_2 \equiv \texttt{App} \, t_3 \, t_2}{t_1 \equiv t_3}\]
  \[\infer[\text{conversion-app-2}]{\texttt{App} \, t_1 \, t_2 \equiv \texttt{App} \, t_1 \, t_3}{t_2 \equiv t_3}\]
\end{itembox}

\(\alpha , \beta\) 変換はこの同値と非常にうまくいくことが(定理としてであって定義ではなく)わかる。
conversion-fun は関数型言語で作るときには人によっては使わないかも(スタックマシンへのコンパイルの場合は、値としての関数は変に reduction せず写すなどするため?)
また、これに
\[\infer[\text{computation-eta}]{t \equiv \texttt{Fun} \, x \, (\texttt{App} \, t \, x)}{}\]
を付け加えることもあるが、外延的ラムダ計算という別の体系になる。