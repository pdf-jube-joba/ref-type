次のような性質を持つ型理論が欲しい。
\begin{itemize}
  \item より自然に property に関する subtyping が使える
  \begin{itemize}
    \item \(2\) が自然数でもあり偶数でもある。
    Coq の場合は \(2\) と \(2\) が偶数であることの証明の組が偶数として型付けされる。
    \item 部分集合が本当に部分集合になり、キャストが簡単(書かなくていい)
    \item 結果として型付けの一意性はないと思うけど、それでもいい
  \end{itemize}

  \item 証明項を真に区別する必要がない or 証明項を扱うことができない
  \begin{itemize}
    \item 群が等しいとは群の演算が等しいこと、証明項まで等しいこととみなしたくない
    \item 証明項を構成することもできるが、それの存在を覚えておくだけぐらいでいい
    \item あと関数の外延性などの axiom をいい感じにしたい
  \end{itemize}

  \item 構造に関する部分型(?)も使えると楽
  \begin{itemize}
    \item 環は群の部分型とみなしたい(キャストを明示的に書きたくない)
    \item これをやると部分空間の扱いが絶対にめんどくさい
    \item 公称型みたいな感じで扱った方がいいかも
  \end{itemize}

  \item 等式をもっと簡単に扱いたい、 well-definedness をもっと簡単に
  \begin{itemize}
    \item 例として、商群からの写像の扱いが Coq ではめんどくさい
    \item (部分集合系が扱えると良いなあ)
  \end{itemize}

\end{itemize}