もうちょっと \(\beta\) 同値を明示的に扱った方が、 eta-conversion とかつけやすい気がした。
また、 ( definitional な?) functional extensionality を最終的に扱うにあたり、何か制限を書けた方がいいかも。
例えば、 \(f_1 , f_2\) が共に \(A \to B , \texttt{Ref} \, A \, P \to \texttt{Ref} \, B \, Q\) と型付けされるとすると、同値であることを \(\texttt{Ref} \, A \, P \to \texttt{Ref} \, B \, Q\) の中で示したら、 \(A \to B\) の中で同値になってしまわないように気を付ける必要がある。
(あるいは \(\texttt{Ref} \, A \, P \to \texttt{Ref} \, B \, Q\) での同値が \(A \to B\) での同値としてもよいレベルの制限?)
今回は \(\beta\) 同値と \(\eta\) 同値を見たことある形で適当に rule に入れることにした。
これは rule にしなくても \(\equiv\) という関係を別に定義して条件として入れてしまってもよい。
あといるのかどうかよくわからなかったので、 Context Start と Axiom を制限することにした。
実装してから困ったら変える。

また、 \(\texttt{Hold}\) に対応するコンストラクタとして \(\texttt{If}\) を導入する方がやりやすくなりそう。

結果として、 equivalenve を明示的に書いただけになってしまった。
あとで well-defined な構成を行うときに使うと思う。
例えば、 \(\texttt{Take} \, x \, A \, t\) を \(A\) が inhabitants で \(A\) の取り方によらないことが示せたら、 \(t\) の型 \(T\) として、 \(\texttt{Take} \, x \, A \, t\) を \(T\) に型付けられるようにしたら、商集合からの写像がうまくいくと思う。

\paragraph*{項やコンテキストの定義}
\begin{grammar}
<term> ::= <variable> 
\alt `Prop'
\alt `Type'
\alt `Fun' <variable> <term> <term>
\alt `For' <variable> <term> <term>
\alt `App' <term> <term>
\alt `Ref' <term> <term>
\alt `If' <term> <term>
\alt `Prf' <term>

<context-snippet> ::= <variable> `:' <term>
\alt Hold <term>

<context> ::= `empty' | <context> `,' <context-snippet>
\end{grammar}

\paragraph*{項やコンテキストの評価}
ここでコンテキストや項の関係を定義していく。
新しく項の同一性を rule に含める。
そのため、木の種類が増える。

\begin{itembox}[l]{コンテキストと項の関係}
  \(\vdash \Gamma\) , コンテキストの well-def 性 \\
  \(\Gamma \vdash t_1 : t_2\) , 項の型付け性 \\
  \(\Gamma \vDash t\) , 項の証明可能性 \\
  \(t_1 \equiv t_2\) , 項の同一性
\end{itembox}

\begin{itembox}[l]{コンテキストの well-def}
  \[\infer[\text{context empty}]{\vdash empty}{}\]
  \[\infer[\text{context start}]{\vdash \Gamma , x : A}{\vdash \Gamma & \Gamma \vdash A : \texttt{Type} & x \notin \Gamma}\]
  \[\infer[\text{context prop}]{\vdash \Gamma , \text{Hold} \, P}{\vdash \Gamma & \Gamma \vdash P : \texttt{Prop}}\]
\end{itembox}

\begin{itembox}[l]{コンテキスト 自明}
  \[\infer[\text{axiom Type}]{empty \vdash \texttt{Sort} : \texttt{Type}}{}\]
  \[\infer[\text{axiom Prop}]{empty \vdash \texttt{Prop} : \texttt{Prop}}{}\]
  \[\infer[\text{variable}]{\Gamma , x : A \vdash x : A}{\vdash \Gamma & \Gamma \vdash A : \texttt{Type} & x \notin \Gamma}\]
  \[\infer[\text{weakning}]{\Gamma , \_ \vdash t : A_2}{\vdash \Gamma , \_ & \Gamma \vdash t : A_2} \]
\end{itembox}

\begin{itembox}[l]{formation}
  \[\infer[\text{forall formation}]{\Gamma \vdash \texttt{For} \, x \, A_1 \, A_2 : \texttt{Sort}_2}{\Gamma \vdash A_1 : \texttt{Sort}_1 & \Gamma , x : A_1 \vdash A_2 : \texttt{Sort}_2 }\]
  \[\infer[\text{refinement formation}]{\Gamma \vdash \texttt{Ref} \, A \, P : \texttt{Type}}{\Gamma \vdash A : \texttt{Type} & \Gamma \vdash P : \texttt{For} \, x \, A \, \texttt{Prop}}\]
\end{itembox}

\begin{itembox}[l]{type の conversion}
  \[\infer[\text{type conversion}]{\Gamma \vdash x : A_2}{\Gamma \vdash x : A_1 & A_1 \equiv A_2 & \Gamma \vdash A_2 : \texttt{Sort}} \]
\end{itembox}

\begin{itembox}[l]{introduction と elimination}
  \[\infer[\text{for intro}]{\Gamma \vdash \texttt{Fun} \, x \, A_1 \, t : \texttt{For} \, x \, A_1 \, A_2}{\Gamma \vdash \texttt{For} \, x \, A_1 \, A_2 : \texttt{Type} & \Gamma , x : A_1 \vdash t : A_2}\]
  \[\infer[\text{for elim}]{\Gamma \vdash \texttt{App} \, t_1 \, t_2 : A_2\{x \leftarrow t_2\}}{\Gamma \vdash t_1 : \texttt{For} \, x \, A_1 \, A_2 & \Gamma \vdash t_2 : A_1}\]
  \[\infer[\text{ref intro}]{\Gamma \vdash t : \texttt{Ref} \, A \, P}{\Gamma \vdash t : A & \Gamma \vdash \texttt{Ref} \, A \, P : \texttt{Type} & \Gamma \vDash \texttt{App} \, P \, t} \]
  \[\infer[\text{ref elim}]{\Gamma \vdash t : A}{\Gamma \vdash t : \texttt{Ref} \, A \, P} \]
  \[\infer[\text{intro}]{\Gamma \vdash \texttt{If} \, P_1 \, p_2 : \texttt{For} \, x \, P_1 \, P_2}{\Gamma \vdash \texttt{For} \, x \, P_1 \, P_2 & \Gamma , \text{Hold} \, P_1 \vdash p_2 : P_2}\]
\end{itembox}

\begin{itembox}[l]{equiv rel}
  \[\infer[\text{alpha-refl}]{t_1 \equiv t_2}{t_1 \equiv_{\alpha} t_2}\]
  \[\infer[\text{sym}]{t_1 \equiv t_2}{t_2 \equiv t_1}\]
  \[\infer[\text{trans}]{t_1 \equiv t_3}{t_1 \equiv t_2 & t_2 \equiv t_3}\]
\end{itembox}

\begin{itembox}[l]{conversion}
  \[\infer[\text{for conversion}]{\texttt{for} \, x \, A^1_1 \, A^1_2 \equiv \texttt{for} \, x \, A^2_1 \, A^2_2}{A^1_1 \equiv A^2_1 & A^1_2 \equiv A^2_2}\]
  \[\infer[\text{fun conversion}]{\texttt{fun} \, x \, t^1_1 \, t^1_2 \equiv \texttt{fun} x \, t^2_1 \, t^2_2}{t^1_1 \equiv t^2_1 & t^1_2 \equiv t^2_2}\]
  \[\infer[\text{app conversion}]{\texttt{app} \, t^1_1 \, t^1_2 \equiv \texttt{app} \, t^2_1 \, t^2_2}{t^1_1 \equiv t^2_1 & t^1_2 \equiv t^2_2}\]
  \[\infer[\text{ref conversion}]{\texttt{ref} \, A^1_1 \, A^1_2 \equiv \texttt{ref} \, A^2_1 \, A^2_2}{A^1_1 \equiv A^2_1 & A^1_2 \equiv A^2_2}\]
\end{itembox}

\begin{itembox}[l]{computation と eta}
  \[\infer[\text{for computation}]{\Gamma \vdash \texttt{app} \, (\texttt{fun} \, x \, A \, t_1) t_2 \equiv t_1\{x \leftarrow t_2\} }{}\]
  \[\infer[\text{for eta}]{\Gamma \vdash \texttt{fun} \, x \, A \, (\texttt{app} \, f \, x) \equiv f}{}\]
\end{itembox}

\begin{itembox}[l]{proof term について}
  \[\infer[\text{assumption}]{\Gamma , \text{Hold} \, P \vDash P}{\vdash \Gamma , \text{Hold} \, P}\]
  \[\infer[\text{weakning}]{\Gamma , \_ \vDash P}{\Gamma \vDash P}\]
  \[\infer[\text{implicit proof}]{\Gamma \vDash P}{\Gamma \vdash P : \texttt{Prop} & \Gamma \vdash t : P} \]
  \[\infer[\text{explicit proof}]{\Gamma \vdash \texttt{Prf} \, P : P}{\Gamma \vDash P} \]
  \[\infer[\text{refinement inversion}]{\Gamma \vDash \texttt{App} \, P \, t}{\Gamma \vdash t : \texttt{Ref} \, A \, P}\]
\end{itembox}