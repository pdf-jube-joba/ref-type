ラムダ計算に型をつける。
このとき、型に加えて新しくコンテキストというものが定義される。
\begin{itembox}[l]{項、型、コンテキストの定義}
\begin{grammar}
  <type ; T> ::= <type-variable ; X>
  \alt `Arr' <type> <type>

  <term ; t> ::= <term-variable ; x>
  \alt `Fun' <variable> <type> <term>
  \alt `App' <term> <term>

  <context ; \(\Gamma\) > ::= empty | <context> ,  <term-variable> : <type>
\end{grammar}
\end{itembox}

項に対しては相変わらず単純に代入が定義された。
また、型があるけれども \(\beta\) 変換やらは普通に定義される。
項の上の \(\equiv\) も同じように定義してしまって構わない。
コピペで持ってくる。
一方で型付けと言う \(\Gamma \vdash t : T\) なる新しい要素が増えた。

\begin{itembox}[l]{型付け規則}
  \[
    \infer[\text{variable}]{\Gamma , x : T \vdash x : T}{} \,
    \infer[\text{weakning}]{\Gamma , x_1 : T_1 \vdash x_2 : T_2}{\Gamma \vdash x_2 : T_2}\]
  \[
    \infer[\text{introduction}]{\Gamma \vdash \texttt{Fun} \, x \, T_1 \, t : \texttt{Arr} \, T_1 \, T_2}{\Gamma , x : T_1 \vdash t : T_2} \,
    \infer[\text{elimination}]{\Gamma \vdash \texttt{App} \, t_1 \, t_2 : T_2}{\Gamma \vdash t_1 : \texttt{Arr} \, T_1 \, T_2 & \Gamma \vdash t_2 : T_1}\]
\end{itembox}

\begin{itembox}[l]{ラムダ項の等式}
  \[
    \infer[\text{alpha}]{\texttt{Fun} \, x_1 \, t_1 \equiv \texttt{Fun} \, x_2 \, (t_1\{x_1 \leftarrow t_2\})}{} \,
    \infer[\text{computation}]{\texttt{App} \, (\texttt{Fun} \, x \, t_1) t_2 \equiv t_2\{x \leftarrow t_2\}}{}\]
  \[
    \infer[\text{conversion-intro}]{\texttt{Fun} \, x \, t_1 \equiv \texttt{Fun} \, x \, t_2}{t_1 \equiv t_2} \,
    \infer[\text{conversion-elim}]{\texttt{App} \, t_1 \, t_2 \equiv \texttt{App} \, t_3 \, t_4}{t_1 \equiv t_3 & t_2 \equiv t_4} \]
\end{itembox}

この体系はいい意味で強くて、型保存性や強正規化定理などがある。
おそらく「 \(\Gamma \vdash t_1 : T , t_1 \equiv t_2\) なら \(\Gamma \vdash t_2 : T\) 」が成り立ちそうだ。
これが成り立つ場合なら、同値性と型付け可能性は別々に定義してしまって構わないと思ってよいが、後でこの二つが強く関係してくるような体系ができる。

type-check は decidable らしい。
型を計算することもできて、 \(F(\Gamma , t)\) なる (option 型) を返す関数が作れて \(F(\Gamma , t) : \text{Some} \, T\) なら \(\Gamma \vdash t : T\) であり、そうでなければ \(\Gamma \vdash t : T\) なる \(T\) がない?