refinement type を普通の依存型に導入し、項が存在するかどうかで refinement type を構成する。 

\paragraph*{項やコンテキストの定義}
\begin{grammar}
<term> ::= <variable> 
\alt `Prop'
\alt `Type'
\alt `Fun' <variable> <term> <term>
\alt `For' <variable> <term> <term>
\alt `App' <term> <term>
\alt `Ref' <term> <term>

<context> ::= empty | <context> `,' <variable> `:' <term>
\end{grammar}

\paragraph*{項やコンテキストの評価}
ここでコンテキストや項の関係を定義していく。

\begin{itembox}[l]{コンテキストと項の関係}
  \(\vdash \Gamma\) , コンテキストの well-def 性 \\
  \(\Gamma \vdash t_1 : t_2\) , 項の型付け性 \\
  \(\Gamma \vDash t\) , 項の証明可能性
\end{itembox}

\begin{itembox}[l]{コンテキストの well-def}
  \[\infer[\text{context empty}]{\vdash empty}{}\]
  \[\infer[\text{well formed}]{\vdash \Gamma , x : A}{\Gamma \vdash A : \texttt{Sort} & x \notin \text{FV}(\Gamma)}\]
\end{itembox}

\begin{itembox}[l]{自然な型付け}
  \[\infer[\text{axiom}]{empty \vdash \texttt{Sort}_1 : \texttt{Sort}_2}{}\]
  \[\infer[\text{variable}]{\Gamma , x : A \vdash x : A}{\Gamma \vdash A : \texttt{Sort} & x \notin \text{FV}(\Gamma)}\]
  \[\infer[\text{weakning}]{\Gamma , x : A \vdash t : A_2}{\Gamma \vdash A : \texttt{Sort} & \Gamma \vdash t : A_2 & x \notin \text{FV}(\Gamma)} \]
\end{itembox}

\begin{itembox}[l]{formation}
  \[\infer[\text{forall formation}]{\Gamma \vdash \texttt{For} \, x \, A_1 \, A_2 : \texttt{Sort}_2}{\Gamma \vdash A_1 : \texttt{Sort}_1 & \Gamma , x : A_1 \vdash A_2 : \texttt{Sort}_2 }\]
  \[\infer[\text{refinement formation}]{\Gamma \vdash \texttt{Ref} \, A \, P : \texttt{Type}}{\Gamma \vdash A : \texttt{Type} & \Gamma \vdash P : \texttt{For} \, x \, A \, \texttt{Prop}}\]
\end{itembox}

\begin{itembox}[l]{introduction と elimination}
  \[\infer[\text{for intro}]{\Gamma \vdash \texttt{Fun} \, x \, A_1 \, t : \texttt{For} \, x \, A_1 \, A_2}{\Gamma , x : A_1 \vdash t : A_2 & \Gamma \vdash \texttt{For} \, x \, A_1 \, A_2 : \texttt{Type}} \]
  \[\infer[\text{for elim}]{\Gamma \vdash \texttt{App} \, t_1 \, t_2 : A_2\{x \leftarrow t_2\}}{\Gamma \vdash t_1 : \texttt{For} \, x \, A_1 \, A_2 & \Gamma \vdash t_2 : A_1} \]
  \[\infer[\text{ref intro}]{\Gamma \vdash t : \texttt{Ref} \, A \, P}{\Gamma \vdash t : A & \Gamma \vdash \texttt{Ref} \, A \, P : \texttt{Type} & \Gamma \vDash \texttt{App} \, P \, t} \]
  \[\infer[\text{ref elim}]{\Gamma \vdash t : A}{\Gamma \vdash t : \texttt{Ref} \, A \, P} \]
\end{itembox}

\begin{itembox}[l]{\(\beta\) 同値について}
  \[\infer[\text{conversion}]{\Gamma \vdash x : A_2}{\Gamma \vdash x : A_1 & A_1 \equiv_{\beta} A_2 & \Gamma \vdash A_2 : \texttt{Sort}} \]
\end{itembox}

\begin{itembox}[l]{証明について}
  \[\infer[\text{inhabitants}]{\Gamma \vDash P}{\Gamma \vdash t : P & \Gamma \vdash P : \texttt{Prop}}\]
  \[\infer[\text{refinement inversion}]{\Gamma \vDash \texttt{App} \, P \, t}{\Gamma \vdash t : \texttt{Ref} \, A \, P}\]
\end{itembox}

失敗している理由は、「型が inhabitants かどうか」という新しい判断だけ追加しているが、 \(\Gamma \vDash P\) から対応する証明項を取り出すことができないため、成り立ってほしい判断が示せなくなったため。

例えば体系に自然数などを追加する。
\(\vDash \text{add} : \texttt{Ref} \, \mathbb{N} \, \text{even} \to \texttt{Ref} \, \mathbb{N} \, \text{even} \to \texttt{Ref} \, \mathbb{N} \, \text{even}\) を示したいときに、
\(n : \texttt{Ref} \, \mathbb{N} \, \text{even} , m : \texttt{Ref} \, \mathbb{N} \, \text{even} \vDash ((\text{add} \, m) \, n) : \texttt{Ref} \, \mathbb{N} \, \text{even}\) を示したいが、 \(((\text{add} \, m) \, n) : \texttt{Ref} \, \mathbb{N} \, \text{even}\) の証明項の構成には \(n , m : \texttt{Ref} \, \mathbb{N} \, \text{even}\) の証明項を取り出したくなるが、できない。