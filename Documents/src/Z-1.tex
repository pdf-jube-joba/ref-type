ただ単に証明木 + refinement type を扱えるようにしただけの体系を考える。
subtyping はない。
証明項というものを完全になくそうとした場合、証明を組むのは証明項ではなく証明木を作ることになる。
問題点としては、もしシステムとして実装するなら証明木を対話的に組んでいく必要があってよりめんどくさそう。
なので、この方向は使わない。
(うまくいかないことがあっても仕方ない。)
一応書いておいただけ。

\paragraph*{項やコンテキストの定義}
\begin{grammar}
<term;t,A,P> ::= <variable;x> 
\alt `Prop'
\alt `Type'
\alt `Fun' <variable> <term> <term>
\alt `For' <variable> <term> <term>
\alt `App' <term> <term>
\alt `Ref' <term> <term>

<context-snippet> ::= <variable> `:' <term>
\alt Hold <term>

<context;\(\Gamma\)> ::= `empty' | <context> `,' <context-snippet>
\end{grammar}

\texttt{Ref} が refinement type の型。
コンテキストの \(\text{Hold} \, \langle term \rangle\) は命題の仮定を表す。
\(t\) と \(A\) は同じ項だが、気持ちとしては項と型の分け方である。
また \(\texttt{Sort}\) は \texttt{Prop} か \texttt{Type} を表す。
自由変数とか subst とかはめんどくさいので書いてない。 conversion rule も定義していないが、 \(M_1 \equiv_{\beta} M_2\) みたいなのが定義されていてほしい。

コンテキストと項の上の関係( rule )を定める。
直観的にはコンテキストの well-def 性、型付け可能性、証明可能、を表す。

\begin{itembox}[l]{rule 一覧}
  \(\vdash \Gamma \) \\
  \(\Gamma \vdash t_1 : t_2\) \\
  \(\Gamma \vDash t \)
\end{itembox}

「コンテキストに含まれる自由変数」などは定義してないが、いい感じに定義されているとする。
ここでは \(\beta\) 同値は rule に含まれない。
これらの rule に関して以下のように rule の間の関係を定める。
(ここで rule と異なるものが並んでいる場合は、ここではそれを条件とでも呼ぶこともある。
例えば、 \(\beta\) 同値は条件である。ただし標準的な呼び方かはわからない。)
( Type : Type を避けるためにいろいろ書いているが無駄だったかもしれない。)

\begin{itembox}[l]{コンテキストの well-def}
  \[\infer[\text{context empty}]{\vdash empty}{} \]
  \[\infer[\text{context type}]{\vdash \Gamma , x : \texttt{Type}}{\vdash \Gamma}\]
  \[\infer[\text{context term}]{\vdash \Gamma , x : A}{\vdash \Gamma & \Gamma \vdash A : \texttt{Type} & x \notin \Gamma} \]
  \[\infer[\text{context prop}]{\vdash \Gamma , \text{Hold} \, P}{\vdash \Gamma & \Gamma \vdash P : \texttt{Prop}} \]
\end{itembox}

\begin{itembox}[l]{コンテキストから自明}
  \[\infer[\text{start}]{\Gamma , x : A \vdash x : A}{\vdash \Gamma & \Gamma \vdash A : \texttt{Type}} \]
  \[\infer[\text{weakning}]{\Gamma , x : A_1 \vdash t : A_2}{\vdash \Gamma , x : A_1 & \Gamma \vdash t : A_2} \]
\end{itembox}

\begin{itembox}[l]{型付け (form) }
  \[\infer[\text{forall formation(type)}]{\Gamma \vdash \texttt{For} \, x \, A_1 \, A_2 : \texttt{Type}}{\Gamma \vdash A_1 : \texttt{Type} & \Gamma , x : A_1 \vdash A_2 : \texttt{Type} } \]
  \[\infer[\text{forall formation(prop)}]{\Gamma \vdash \texttt{For} \, x \, A_1 \, A_2 : \texttt{Prop}}{\Gamma \vdash A_1 : \texttt{Type} & \Gamma , x : A_1 \vdash A_2 : \texttt{Prop} } \]
  \[\infer[\text{forall formation(prop2)}]{\Gamma \vdash \texttt{For} \, x \, A_1 \, A_2 : \texttt{Prop}}{\Gamma \vdash A_1 : \texttt{Type} & \Gamma , x : A_1 \vdash A_2 : \texttt{Prop} }\]
  \[\infer[\text{refinement formation}]{\Gamma \vdash \texttt{Ref} \, A \, P : \texttt{Type}}{\Gamma \vdash A : \texttt{Type} & \Gamma \vdash P : \texttt{For} \, x \, A \, \texttt{Prop}} \]
  Prop から Type を作るのは禁止したいので型付けられないようにしてある
\end{itembox}

\begin{itembox}[l]{conversion}
  \[\infer[\text{conversion(type)}]{\Gamma \vdash x : A_2}{\Gamma \vdash x : A_1 & A_1 \equiv_{\beta} A_2 & \Gamma \vdash A_2 : \texttt{Type}} \]
  \[\infer[\text{(type)}]{\Gamma \vdash A_2 : \texttt{Type}}{\Gamma \vdash A_1 : \texttt{Type} & A_1 \equiv_{\beta} A_2} \]
  \[\infer[\text{(prop)}]{\Gamma \vdash A_2 : \texttt{Prop}}{\Gamma \vdash A_1 : \texttt{Prop} & A_1 \equiv_{\beta} A_2} \]
  application や conversion により 下二つはいらないかもしれないし、逆に強すぎるかもしれない。
  ( subject reduction (?) がうまくいくのかわからなかったのでとりあえずつけた。 ITT での equality に関する judgement みたいなのがないのでバグがありそう)
  変な問題があっても直す気はない。
\end{itembox}

\begin{itembox}[l]{型付け (intro と elim) }
  \[\infer[\text{forall introduction}]{\Gamma \vdash \texttt{Fun} \, x \, A_1 \, t : \texttt{For} \, x \, A_1 \, A_2}{\Gamma , x : A_1 \vdash t : A_2 & \Gamma \vdash \texttt{For} \, x \, A_1 \, A_2 : \texttt{Type}} \]
  \[\infer[\text{forall elimination}]{\Gamma \vdash \texttt{App} \, t_1 \, t_2 : A_2\{x \leftarrow t_2\}}{\Gamma \vdash t_1 : \texttt{For} \, x \, A_1 \, A_2 & \Gamma \vdash t_2 : A_1} \]
  \[\infer[\text{refinement introduction}]{\Gamma \vdash t : \texttt{Ref} \, A \, P}{\Gamma \vdash t : A & \Gamma \vdash \texttt{Ref} \, A \, P : \texttt{Type} & \Gamma \vDash \texttt{App} \, P \, t} \]
  \[\infer[\text{refinement elimination}]{\Gamma \vdash t : A}{\Gamma \vdash t : \texttt{Ref} \, A \, P} \]
\end{itembox}

\begin{itembox}[l]{Proof}
  \[\infer[\text{prop conversion}]{\Gamma \vDash P_2}{\Gamma \vdash P_1 : \texttt{Prop} & P_1 \equiv_{\beta} P_2 & \Gamma \vDash P_1} \]
  \[\infer[\text{forall intro(prop)}]{\Gamma \vDash \texttt{For} \, \, x \, A \, P}{\Gamma \vdash \texttt{For} \, x \, A \, P : \texttt{Prop} & \Gamma , x : A \vDash \texttt{App} \, P \, x} \]
  \[\infer[\text{forall elim(prop)}]{\Gamma \vDash P\{x \leftarrow t\}}{\Gamma \vDash \texttt{For} \, x \, A \, P & \Gamma \vdash t : A} \]
  \[\infer[\text{refinement elim(prop)}]{\Gamma \vDash \texttt{App} \, P \, x}{\Gamma \vdash x : \texttt{Ref} \, A \, P} \]
\end{itembox}

これによくある論理の何某を付け加えて終わり。
この型理論を検討することはないと思うが、例を後で挙げたい。
Prop と Type で別々の型付けが必要になることがありめんどくさい。