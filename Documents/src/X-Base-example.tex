具体例を出したい。
\begin{itemize}
  \item \(A : \texttt{Prop} , B : \texttt{Prop} , \text{Hold} \, (\texttt{for} \, x \, A \, B) , \text{Hold} \, A \vDash B\)
  \item \(A , B : \texttt{Type}\) , \(f : A \to B\) とする。\(P , Q : A , B\) 上の述語に対して 「任意の \(x : A\) について、 \(P(x)\) が成り立つなら \(Q(f(x))\) が成り立つ」とする。
  このとき \(f : \texttt{Ref} \, A \, P \to \texttt{Ref} B \, Q \) と型付けられるはず。
\end{itemize}
一つ目は省略(カリーハワードと Proof を使う)
コンテキストとしては、
  \[\Gamma := A : \texttt{Type} , B : \texttt{Type} , f : A \to B , P : A \to \texttt{Prop} , Q : B \to \texttt{Prop}\]
をまず考え、これに命題に対応する項がくっつく。
\(K := \texttt{For} \, x \, A \, ((\texttt{App} \, P \, x) \to (\texttt{App} \, Q \, (\texttt{App} \, f \, x))))\) なる \(\texttt{Prop}\) 型の項が 「任意の \(x : A\) について、 \(P(x)\) が成り立つなら \(\ldots\) 」に対応する。
改めて \(\Gamma \leftarrow \Gamma , p : K\) と置き直して、 \(\Gamma \vdash f : \texttt{Ref} \, A \, P \to \texttt{Ref} \, B \, Q\) を示す。
eta-conversion が必要になっちゃった。
\[
  \infer[\text{eta-conversion}]
  {\Gamma \vdash f : \texttt{For} \, x \, (\texttt{Ref} \, A \, P) \, (\texttt{Ref} \, B \, Q)}{
    \infer[\text{for intro}]
    {\Gamma \vdash \texttt{Fun} \, x \, (\texttt{Ref} \, A \, P) \, (\texttt{App} \, f \, x) : \texttt{For} \, x \, (\texttt{Ref} \, A \, P) \, (\texttt{Ref} \, B \, Q)}{
      \infer{\Gamma \vdash \texttt{For} \, x \, (\texttt{Ref} \, A \, P) \, (\texttt{Ref} \, B \, Q) : \texttt{Type}}{\vdots} &
      \infer{{\Gamma_1 := \Gamma , x : \texttt{Ref} \, A \, P \vdash \texttt{App} \, f \, x : \texttt{Ref} \, B \, Q}}{\vdots}
  }}
\]
左側は仮定を思い出せばよい。
\[\infer[\text{for form}]{\Gamma \vdash \texttt{For} \, x \, (\texttt{Ref} \, A \, P) \, (\texttt{Ref} \, B \, Q) : \texttt{Type}}{
  \infer[\text{ref form}]{\Gamma \vdash \texttt{Ref} \, A \, P : \texttt{Type}}
  {
    \infer[\text{var}]{\Gamma \vdash A : \texttt{Type}}{} &
    \infer[\text{var}]{\Gamma \vdash P : \texttt{For} \, x \, A \, \texttt{Prop}}{}
  } &
  \infer[]{\Gamma , x : \texttt{Ref} \, A \, P \vdash \texttt{Ref} \, B \, Q : \texttt{Type}}{\vdots}
}\]
右側は
\[\infer[]
  {\Gamma_1 \vdash \texttt{App} \, f \, x : \texttt{Ref} \, B \, Q}{
  \infer[\text{for intro}]
  {\Gamma_1 \vdash \texttt{App} \, f \, x : B}{
    \Gamma_1 \vdash f : A \to B & \Gamma_1 \vdash x : A} &
  \infer[\text{var}]
  {\Gamma_1 \vdash \texttt{Ref} \, B \, Q : \texttt{Type} }{\vdots} &
  \infer[X_1]
  {\Gamma_1 \vDash \texttt{App} \, Q \, (\texttt{App} \, f \, x)}{
  \vdots
  }
}\]
\(X_1\) は app 省略して
\[
  \infer[\text{imp prf}]
  {\Gamma_1 \vDash \texttt{App} \, Q \, (\texttt{App} \, f \, x)}{
    \infer[\text{for elim}]
    {\Gamma_1 \vdash (t \, x) \, (\texttt{Prf} \, (P \, x)) : \texttt{App} \, Q \, (\texttt{App} \, f \, x)}{
      \infer[\text{for elim}]{\Gamma_1 \vdash (t \, x) : (P \, x) \to (Q \, (f \, x))}{
        \infer[\text{imp prf}]{\Gamma_1 \vdash t : K}{
          \infer[\text{assumption}]{\Gamma_1 \vDash K}{}}
      } &
      \infer[\text{exp prf}]{\Gamma_1 \vdash \texttt{Prf} \, (P \, x) : P \, x}{
        \infer[\text{ref inv}]{\Gamma_1 \vDash P \, x}{
          \infer[\text{var term}]{\Gamma_1 \vdash x : \texttt{Ref} \, A \, P}{}}
    }
    }
  }
\]
\(\_ \vDash \_ \) が必要になったときのほとんどは自動的に推論できそう