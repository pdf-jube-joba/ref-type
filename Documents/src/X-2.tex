文法周りは以下。

\paragraph*{項やコンテキストの定義}
\begin{grammar}
<term> ::= <variable> 
\alt `Prop'
\alt `Type'
\alt `Fun' <variable> <term> <term>
\alt `For' <variable> <term> <term>
\alt `App' <term> <term>
\alt `Ref' <term> <term>
\alt `Prf' <term>

<context-snippet> ::= <variable> `:' <term> | Hold <term>

<context> ::= `empty' | <context> `,' <context-snippet>
\end{grammar}

\texttt{Ref} が refinement type の型。
\(\texttt{Ref} \, A \, P\) で型 \(A\) を述語 \(P\) : \(A \to \texttt{Prop}\) で refine した型を表す。
項が refinement type に型付けされるときには述語 \(P\) が満たされているかどうか、つまり inhabitants かどうかを解くことになる。

\texttt{Prf} は証明項を陽に扱うためのもの。
命題 \(P\) が示せるときに \(\texttt{Prf} \, P\) を証明項として扱ってよい。

\paragraph*{項やコンテキストの評価}
ここでコンテキストや項の関係を定義していく。

\begin{itembox}[l]{コンテキストと項の関係}
  \(\vdash \Gamma\) , コンテキストの well-def 性 \\
  \(\Gamma \vdash t_1 : t_2\) , 項の型付け性 \\
  \(\Gamma \vDash t\) , 項の証明可能性
\end{itembox}

判断のなかに項の証明可能性を含めて定義するのは、
どこで証明項の存在が要求されているかわかりやすくするため。
(もしかしたらそうしない方が簡単になるかもしれない?)

\begin{itembox}[l]{コンテキストの well-formed}
  \[\infer[\text{context empty}]{\vdash empty}{}\]
  \[\infer[\text{context start}]{\vdash \Gamma , x : A}{\Gamma \vdash A : \texttt{Type} & x \notin \text{FV}(\Gamma)}\]
  \[\infer[\text{context prop}]{\vdash \Gamma , \text{Hold} \, P}{\Gamma \vdash P : \texttt{Prop}}\]
\end{itembox}

\begin{itembox}[l]{自然な型付け}
  \[\infer[\text{axiom}]{empty \vdash \texttt{Sort}_1 : \texttt{Sort}_2}{}\]
  \[\infer[\text{variable}]{\Gamma , x : A \vdash x : A}{\vdash \Gamma & \Gamma \vdash A : \texttt{Type} & x \notin \text{FV}(\Gamma)}\]
  \[\infer[\text{weakning}]{\Gamma , \_ \vdash t : A_2}{\vdash \Gamma , \_ & \Gamma \vdash t : A_2}\]
\end{itembox}

\begin{itembox}[l]{formation}
  \[\infer[\text{forall formation}]{\Gamma \vdash \texttt{For} \, x \, A_1 \, A_2 : \texttt{Sort}_2}{\Gamma \vdash A_1 : \texttt{Sort}_1 & \Gamma , x : A_1 \vdash A_2 : \texttt{Sort}_2 }\]
  \[\infer[\text{refinement formation}]{\Gamma \vdash \texttt{Ref} \, A \, P : \texttt{Type}}{\Gamma \vdash A : \texttt{Type} & \Gamma \vdash P : \texttt{For} \, x \, A \, \texttt{Prop}}\]
\end{itembox}

\begin{itembox}[l]{introduction と elimination}
  \[\infer[\text{for intro}]{\Gamma \vdash \texttt{Fun} \, x \, A_1 \, t : \texttt{For} \, x \, A_1 \, A_2}{\Gamma , x : A_1 \vdash t : A_2 & \Gamma \vdash \texttt{For} \, x \, A_1 \, A_2 : \texttt{Type}} \]
  \[\infer[\text{for elim}]{\Gamma \vdash \texttt{App} \, t_1 \, t_2 : A_2\{x \leftarrow t_2\}}{\Gamma \vdash t_1 : \texttt{For} \, x \, A_1 \, A_2 & \Gamma \vdash t_2 : A_1} \]
  \[\infer[\text{ref intro}]{\Gamma \vdash t : \texttt{Ref} \, A \, P}{\Gamma \vdash t : A & \Gamma \vdash \texttt{Ref} \, A \, P : \texttt{Type} & \Gamma \vDash \texttt{App} \, P \, t} \]
  \[\infer[\text{ref elim}]{\Gamma \vdash t : A}{\Gamma \vdash t : \texttt{Ref} \, A \, P} \]
\end{itembox}

\begin{itembox}[l]{\(\beta\) 同値について}
  \[\infer[\text{conversion}]{\Gamma \vdash x : A_2}{\Gamma \vdash x : A_1 & A_1 \equiv_{\beta} A_2 & \Gamma \vdash A_2 : \texttt{Sort}} \]
\end{itembox}

\begin{itembox}[l]{proof term について}
  \[\infer[\text{assumption}]{\Gamma , \text{Hold} \, P \vDash P}{}\]
  \[\infer[\text{implicit proof}]{\Gamma \vDash P}{\Gamma \vdash P : \texttt{Prop} & \Gamma \vdash t : P} \]
  \[\infer[\text{explicit proof}]{\Gamma \vdash \texttt{Prf} \, P : P}{\Gamma \vdash P : \texttt{Prop} &\Gamma \vDash P} \]
  \[\infer[\text{refinement inversion}]{\Gamma \vDash \texttt{App} \, P \, t}{\Gamma \vdash t : \texttt{Ref} \, A \, P}\]
\end{itembox}